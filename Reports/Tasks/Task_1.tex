\chapter{Initial organization}
\label{task1}

\section{Objectives}

For this project, three milestones have been defined with the aim of applying Big Data processing and machine learning/artificial intelligence techniques to traffic and weather data.\\

First of all, the team will perform an \underline{analysis} on the Dutch traffic data and the weather information collected. Within that frame, three tasks have been defined:

\begin{enumerate}
	\item \textbf{Filtering}: the data will be collected, filtered and structured in order to facilitate its manipulation and reduce the memory storage required.
	\item \textbf{Statistics}: with the data collected, statistical conclusions will be drawn. Both datasets, the traffic and weather data, will be merge bearing in mind time synchronization. During this stage multiple graphs and tables will be generated.
	\item \textbf{Results}: conclusions will be drawn and documented in a report with a thorough explanation of the visuals created.
\end{enumerate}

Secondly, the group will apply artificial intelligence techniques (such as an Artificial Neural Network) in order to generate a model that could \underline{predict} future traffic jams. The ANN will be trained with historical data and the group aims at implementing the final model in a web service that is fed with live data.\\

Finally, if the group counts with enough time, an \underline{optimization model} will be created. Such a model would interact with a user, in this case a driver, indicating what would be the optimal speed at which he could ride in order to avoid a traffic jam. Another possibility would be pointing the user to an alternative route such that he/she could avoid an expected traffic jam.

\section{Data}

For this assignment, the team requires historical and live data of the Dutch traffic and the weather conditions on the road.\\

For the first one, the group retrieves the information from the Nationale Databank Wegverkeergegevens (\underline{NDW}); the data is collected in an XML format.\\

With respect to the second information required, the weather data is collected from the Koninklijk Nederlands Meteorologisch Instituut (\underline{KNMI}) in a text format.\\

Besides the previous data banks, the schedule of the yearly Dutch holidays will be also included since it will help in the prediction of traffic jams of the AI model.\\

\section{Tools}

The following software tools will be used in order to carry out the project successfully:

\begin{enumerate}
	\item \textbf{Python}: programming language key for the processing of the data.
	\item \textbf{Spark}: open-source cluster-computing framework that will be extensively used for the processing of the data. The Spark software also provides us with a set of components, such as the Spark Streaming, the MLLib and the GraphX; which will help with the data analysis and the creation of the artificial intelligent model.	
	\item \textbf{Jupyter notebook}: web application for the code documentation.
	\item \textbf{GitHub}: development platform to ease the collaboration of the team members to the project's code.
	\item \textbf{Stack}: online server for the storage of high amounts of data.
\end{enumerate}
